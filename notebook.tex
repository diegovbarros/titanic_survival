
% Default to the notebook output style

    


% Inherit from the specified cell style.




    
\documentclass[11pt]{article}

    
    
    \usepackage[T1]{fontenc}
    % Nicer default font (+ math font) than Computer Modern for most use cases
    \usepackage{mathpazo}

    % Basic figure setup, for now with no caption control since it's done
    % automatically by Pandoc (which extracts ![](path) syntax from Markdown).
    \usepackage{graphicx}
    % We will generate all images so they have a width \maxwidth. This means
    % that they will get their normal width if they fit onto the page, but
    % are scaled down if they would overflow the margins.
    \makeatletter
    \def\maxwidth{\ifdim\Gin@nat@width>\linewidth\linewidth
    \else\Gin@nat@width\fi}
    \makeatother
    \let\Oldincludegraphics\includegraphics
    % Set max figure width to be 80% of text width, for now hardcoded.
    \renewcommand{\includegraphics}[1]{\Oldincludegraphics[width=.8\maxwidth]{#1}}
    % Ensure that by default, figures have no caption (until we provide a
    % proper Figure object with a Caption API and a way to capture that
    % in the conversion process - todo).
    \usepackage{caption}
    \DeclareCaptionLabelFormat{nolabel}{}
    \captionsetup{labelformat=nolabel}

    \usepackage{adjustbox} % Used to constrain images to a maximum size 
    \usepackage{xcolor} % Allow colors to be defined
    \usepackage{enumerate} % Needed for markdown enumerations to work
    \usepackage{geometry} % Used to adjust the document margins
    \usepackage{amsmath} % Equations
    \usepackage{amssymb} % Equations
    \usepackage{textcomp} % defines textquotesingle
    % Hack from http://tex.stackexchange.com/a/47451/13684:
    \AtBeginDocument{%
        \def\PYZsq{\textquotesingle}% Upright quotes in Pygmentized code
    }
    \usepackage{upquote} % Upright quotes for verbatim code
    \usepackage{eurosym} % defines \euro
    \usepackage[mathletters]{ucs} % Extended unicode (utf-8) support
    \usepackage[utf8x]{inputenc} % Allow utf-8 characters in the tex document
    \usepackage{fancyvrb} % verbatim replacement that allows latex
    \usepackage{grffile} % extends the file name processing of package graphics 
                         % to support a larger range 
    % The hyperref package gives us a pdf with properly built
    % internal navigation ('pdf bookmarks' for the table of contents,
    % internal cross-reference links, web links for URLs, etc.)
    \usepackage{hyperref}
    \usepackage{longtable} % longtable support required by pandoc >1.10
    \usepackage{booktabs}  % table support for pandoc > 1.12.2
    \usepackage[inline]{enumitem} % IRkernel/repr support (it uses the enumerate* environment)
    \usepackage[normalem]{ulem} % ulem is needed to support strikethroughs (\sout)
                                % normalem makes italics be italics, not underlines
    

    
    
    % Colors for the hyperref package
    \definecolor{urlcolor}{rgb}{0,.145,.698}
    \definecolor{linkcolor}{rgb}{.71,0.21,0.01}
    \definecolor{citecolor}{rgb}{.12,.54,.11}

    % ANSI colors
    \definecolor{ansi-black}{HTML}{3E424D}
    \definecolor{ansi-black-intense}{HTML}{282C36}
    \definecolor{ansi-red}{HTML}{E75C58}
    \definecolor{ansi-red-intense}{HTML}{B22B31}
    \definecolor{ansi-green}{HTML}{00A250}
    \definecolor{ansi-green-intense}{HTML}{007427}
    \definecolor{ansi-yellow}{HTML}{DDB62B}
    \definecolor{ansi-yellow-intense}{HTML}{B27D12}
    \definecolor{ansi-blue}{HTML}{208FFB}
    \definecolor{ansi-blue-intense}{HTML}{0065CA}
    \definecolor{ansi-magenta}{HTML}{D160C4}
    \definecolor{ansi-magenta-intense}{HTML}{A03196}
    \definecolor{ansi-cyan}{HTML}{60C6C8}
    \definecolor{ansi-cyan-intense}{HTML}{258F8F}
    \definecolor{ansi-white}{HTML}{C5C1B4}
    \definecolor{ansi-white-intense}{HTML}{A1A6B2}

    % commands and environments needed by pandoc snippets
    % extracted from the output of `pandoc -s`
    \providecommand{\tightlist}{%
      \setlength{\itemsep}{0pt}\setlength{\parskip}{0pt}}
    \DefineVerbatimEnvironment{Highlighting}{Verbatim}{commandchars=\\\{\}}
    % Add ',fontsize=\small' for more characters per line
    \newenvironment{Shaded}{}{}
    \newcommand{\KeywordTok}[1]{\textcolor[rgb]{0.00,0.44,0.13}{\textbf{{#1}}}}
    \newcommand{\DataTypeTok}[1]{\textcolor[rgb]{0.56,0.13,0.00}{{#1}}}
    \newcommand{\DecValTok}[1]{\textcolor[rgb]{0.25,0.63,0.44}{{#1}}}
    \newcommand{\BaseNTok}[1]{\textcolor[rgb]{0.25,0.63,0.44}{{#1}}}
    \newcommand{\FloatTok}[1]{\textcolor[rgb]{0.25,0.63,0.44}{{#1}}}
    \newcommand{\CharTok}[1]{\textcolor[rgb]{0.25,0.44,0.63}{{#1}}}
    \newcommand{\StringTok}[1]{\textcolor[rgb]{0.25,0.44,0.63}{{#1}}}
    \newcommand{\CommentTok}[1]{\textcolor[rgb]{0.38,0.63,0.69}{\textit{{#1}}}}
    \newcommand{\OtherTok}[1]{\textcolor[rgb]{0.00,0.44,0.13}{{#1}}}
    \newcommand{\AlertTok}[1]{\textcolor[rgb]{1.00,0.00,0.00}{\textbf{{#1}}}}
    \newcommand{\FunctionTok}[1]{\textcolor[rgb]{0.02,0.16,0.49}{{#1}}}
    \newcommand{\RegionMarkerTok}[1]{{#1}}
    \newcommand{\ErrorTok}[1]{\textcolor[rgb]{1.00,0.00,0.00}{\textbf{{#1}}}}
    \newcommand{\NormalTok}[1]{{#1}}
    
    % Additional commands for more recent versions of Pandoc
    \newcommand{\ConstantTok}[1]{\textcolor[rgb]{0.53,0.00,0.00}{{#1}}}
    \newcommand{\SpecialCharTok}[1]{\textcolor[rgb]{0.25,0.44,0.63}{{#1}}}
    \newcommand{\VerbatimStringTok}[1]{\textcolor[rgb]{0.25,0.44,0.63}{{#1}}}
    \newcommand{\SpecialStringTok}[1]{\textcolor[rgb]{0.73,0.40,0.53}{{#1}}}
    \newcommand{\ImportTok}[1]{{#1}}
    \newcommand{\DocumentationTok}[1]{\textcolor[rgb]{0.73,0.13,0.13}{\textit{{#1}}}}
    \newcommand{\AnnotationTok}[1]{\textcolor[rgb]{0.38,0.63,0.69}{\textbf{\textit{{#1}}}}}
    \newcommand{\CommentVarTok}[1]{\textcolor[rgb]{0.38,0.63,0.69}{\textbf{\textit{{#1}}}}}
    \newcommand{\VariableTok}[1]{\textcolor[rgb]{0.10,0.09,0.49}{{#1}}}
    \newcommand{\ControlFlowTok}[1]{\textcolor[rgb]{0.00,0.44,0.13}{\textbf{{#1}}}}
    \newcommand{\OperatorTok}[1]{\textcolor[rgb]{0.40,0.40,0.40}{{#1}}}
    \newcommand{\BuiltInTok}[1]{{#1}}
    \newcommand{\ExtensionTok}[1]{{#1}}
    \newcommand{\PreprocessorTok}[1]{\textcolor[rgb]{0.74,0.48,0.00}{{#1}}}
    \newcommand{\AttributeTok}[1]{\textcolor[rgb]{0.49,0.56,0.16}{{#1}}}
    \newcommand{\InformationTok}[1]{\textcolor[rgb]{0.38,0.63,0.69}{\textbf{\textit{{#1}}}}}
    \newcommand{\WarningTok}[1]{\textcolor[rgb]{0.38,0.63,0.69}{\textbf{\textit{{#1}}}}}
    
    
    % Define a nice break command that doesn't care if a line doesn't already
    % exist.
    \def\br{\hspace*{\fill} \\* }
    % Math Jax compatability definitions
    \def\gt{>}
    \def\lt{<}
    % Document parameters
    \title{Fundamentos de Data Science I - Projeto Final}
    
    
    

    % Pygments definitions
    
\makeatletter
\def\PY@reset{\let\PY@it=\relax \let\PY@bf=\relax%
    \let\PY@ul=\relax \let\PY@tc=\relax%
    \let\PY@bc=\relax \let\PY@ff=\relax}
\def\PY@tok#1{\csname PY@tok@#1\endcsname}
\def\PY@toks#1+{\ifx\relax#1\empty\else%
    \PY@tok{#1}\expandafter\PY@toks\fi}
\def\PY@do#1{\PY@bc{\PY@tc{\PY@ul{%
    \PY@it{\PY@bf{\PY@ff{#1}}}}}}}
\def\PY#1#2{\PY@reset\PY@toks#1+\relax+\PY@do{#2}}

\expandafter\def\csname PY@tok@w\endcsname{\def\PY@tc##1{\textcolor[rgb]{0.73,0.73,0.73}{##1}}}
\expandafter\def\csname PY@tok@c\endcsname{\let\PY@it=\textit\def\PY@tc##1{\textcolor[rgb]{0.25,0.50,0.50}{##1}}}
\expandafter\def\csname PY@tok@cp\endcsname{\def\PY@tc##1{\textcolor[rgb]{0.74,0.48,0.00}{##1}}}
\expandafter\def\csname PY@tok@k\endcsname{\let\PY@bf=\textbf\def\PY@tc##1{\textcolor[rgb]{0.00,0.50,0.00}{##1}}}
\expandafter\def\csname PY@tok@kp\endcsname{\def\PY@tc##1{\textcolor[rgb]{0.00,0.50,0.00}{##1}}}
\expandafter\def\csname PY@tok@kt\endcsname{\def\PY@tc##1{\textcolor[rgb]{0.69,0.00,0.25}{##1}}}
\expandafter\def\csname PY@tok@o\endcsname{\def\PY@tc##1{\textcolor[rgb]{0.40,0.40,0.40}{##1}}}
\expandafter\def\csname PY@tok@ow\endcsname{\let\PY@bf=\textbf\def\PY@tc##1{\textcolor[rgb]{0.67,0.13,1.00}{##1}}}
\expandafter\def\csname PY@tok@nb\endcsname{\def\PY@tc##1{\textcolor[rgb]{0.00,0.50,0.00}{##1}}}
\expandafter\def\csname PY@tok@nf\endcsname{\def\PY@tc##1{\textcolor[rgb]{0.00,0.00,1.00}{##1}}}
\expandafter\def\csname PY@tok@nc\endcsname{\let\PY@bf=\textbf\def\PY@tc##1{\textcolor[rgb]{0.00,0.00,1.00}{##1}}}
\expandafter\def\csname PY@tok@nn\endcsname{\let\PY@bf=\textbf\def\PY@tc##1{\textcolor[rgb]{0.00,0.00,1.00}{##1}}}
\expandafter\def\csname PY@tok@ne\endcsname{\let\PY@bf=\textbf\def\PY@tc##1{\textcolor[rgb]{0.82,0.25,0.23}{##1}}}
\expandafter\def\csname PY@tok@nv\endcsname{\def\PY@tc##1{\textcolor[rgb]{0.10,0.09,0.49}{##1}}}
\expandafter\def\csname PY@tok@no\endcsname{\def\PY@tc##1{\textcolor[rgb]{0.53,0.00,0.00}{##1}}}
\expandafter\def\csname PY@tok@nl\endcsname{\def\PY@tc##1{\textcolor[rgb]{0.63,0.63,0.00}{##1}}}
\expandafter\def\csname PY@tok@ni\endcsname{\let\PY@bf=\textbf\def\PY@tc##1{\textcolor[rgb]{0.60,0.60,0.60}{##1}}}
\expandafter\def\csname PY@tok@na\endcsname{\def\PY@tc##1{\textcolor[rgb]{0.49,0.56,0.16}{##1}}}
\expandafter\def\csname PY@tok@nt\endcsname{\let\PY@bf=\textbf\def\PY@tc##1{\textcolor[rgb]{0.00,0.50,0.00}{##1}}}
\expandafter\def\csname PY@tok@nd\endcsname{\def\PY@tc##1{\textcolor[rgb]{0.67,0.13,1.00}{##1}}}
\expandafter\def\csname PY@tok@s\endcsname{\def\PY@tc##1{\textcolor[rgb]{0.73,0.13,0.13}{##1}}}
\expandafter\def\csname PY@tok@sd\endcsname{\let\PY@it=\textit\def\PY@tc##1{\textcolor[rgb]{0.73,0.13,0.13}{##1}}}
\expandafter\def\csname PY@tok@si\endcsname{\let\PY@bf=\textbf\def\PY@tc##1{\textcolor[rgb]{0.73,0.40,0.53}{##1}}}
\expandafter\def\csname PY@tok@se\endcsname{\let\PY@bf=\textbf\def\PY@tc##1{\textcolor[rgb]{0.73,0.40,0.13}{##1}}}
\expandafter\def\csname PY@tok@sr\endcsname{\def\PY@tc##1{\textcolor[rgb]{0.73,0.40,0.53}{##1}}}
\expandafter\def\csname PY@tok@ss\endcsname{\def\PY@tc##1{\textcolor[rgb]{0.10,0.09,0.49}{##1}}}
\expandafter\def\csname PY@tok@sx\endcsname{\def\PY@tc##1{\textcolor[rgb]{0.00,0.50,0.00}{##1}}}
\expandafter\def\csname PY@tok@m\endcsname{\def\PY@tc##1{\textcolor[rgb]{0.40,0.40,0.40}{##1}}}
\expandafter\def\csname PY@tok@gh\endcsname{\let\PY@bf=\textbf\def\PY@tc##1{\textcolor[rgb]{0.00,0.00,0.50}{##1}}}
\expandafter\def\csname PY@tok@gu\endcsname{\let\PY@bf=\textbf\def\PY@tc##1{\textcolor[rgb]{0.50,0.00,0.50}{##1}}}
\expandafter\def\csname PY@tok@gd\endcsname{\def\PY@tc##1{\textcolor[rgb]{0.63,0.00,0.00}{##1}}}
\expandafter\def\csname PY@tok@gi\endcsname{\def\PY@tc##1{\textcolor[rgb]{0.00,0.63,0.00}{##1}}}
\expandafter\def\csname PY@tok@gr\endcsname{\def\PY@tc##1{\textcolor[rgb]{1.00,0.00,0.00}{##1}}}
\expandafter\def\csname PY@tok@ge\endcsname{\let\PY@it=\textit}
\expandafter\def\csname PY@tok@gs\endcsname{\let\PY@bf=\textbf}
\expandafter\def\csname PY@tok@gp\endcsname{\let\PY@bf=\textbf\def\PY@tc##1{\textcolor[rgb]{0.00,0.00,0.50}{##1}}}
\expandafter\def\csname PY@tok@go\endcsname{\def\PY@tc##1{\textcolor[rgb]{0.53,0.53,0.53}{##1}}}
\expandafter\def\csname PY@tok@gt\endcsname{\def\PY@tc##1{\textcolor[rgb]{0.00,0.27,0.87}{##1}}}
\expandafter\def\csname PY@tok@err\endcsname{\def\PY@bc##1{\setlength{\fboxsep}{0pt}\fcolorbox[rgb]{1.00,0.00,0.00}{1,1,1}{\strut ##1}}}
\expandafter\def\csname PY@tok@kc\endcsname{\let\PY@bf=\textbf\def\PY@tc##1{\textcolor[rgb]{0.00,0.50,0.00}{##1}}}
\expandafter\def\csname PY@tok@kd\endcsname{\let\PY@bf=\textbf\def\PY@tc##1{\textcolor[rgb]{0.00,0.50,0.00}{##1}}}
\expandafter\def\csname PY@tok@kn\endcsname{\let\PY@bf=\textbf\def\PY@tc##1{\textcolor[rgb]{0.00,0.50,0.00}{##1}}}
\expandafter\def\csname PY@tok@kr\endcsname{\let\PY@bf=\textbf\def\PY@tc##1{\textcolor[rgb]{0.00,0.50,0.00}{##1}}}
\expandafter\def\csname PY@tok@bp\endcsname{\def\PY@tc##1{\textcolor[rgb]{0.00,0.50,0.00}{##1}}}
\expandafter\def\csname PY@tok@fm\endcsname{\def\PY@tc##1{\textcolor[rgb]{0.00,0.00,1.00}{##1}}}
\expandafter\def\csname PY@tok@vc\endcsname{\def\PY@tc##1{\textcolor[rgb]{0.10,0.09,0.49}{##1}}}
\expandafter\def\csname PY@tok@vg\endcsname{\def\PY@tc##1{\textcolor[rgb]{0.10,0.09,0.49}{##1}}}
\expandafter\def\csname PY@tok@vi\endcsname{\def\PY@tc##1{\textcolor[rgb]{0.10,0.09,0.49}{##1}}}
\expandafter\def\csname PY@tok@vm\endcsname{\def\PY@tc##1{\textcolor[rgb]{0.10,0.09,0.49}{##1}}}
\expandafter\def\csname PY@tok@sa\endcsname{\def\PY@tc##1{\textcolor[rgb]{0.73,0.13,0.13}{##1}}}
\expandafter\def\csname PY@tok@sb\endcsname{\def\PY@tc##1{\textcolor[rgb]{0.73,0.13,0.13}{##1}}}
\expandafter\def\csname PY@tok@sc\endcsname{\def\PY@tc##1{\textcolor[rgb]{0.73,0.13,0.13}{##1}}}
\expandafter\def\csname PY@tok@dl\endcsname{\def\PY@tc##1{\textcolor[rgb]{0.73,0.13,0.13}{##1}}}
\expandafter\def\csname PY@tok@s2\endcsname{\def\PY@tc##1{\textcolor[rgb]{0.73,0.13,0.13}{##1}}}
\expandafter\def\csname PY@tok@sh\endcsname{\def\PY@tc##1{\textcolor[rgb]{0.73,0.13,0.13}{##1}}}
\expandafter\def\csname PY@tok@s1\endcsname{\def\PY@tc##1{\textcolor[rgb]{0.73,0.13,0.13}{##1}}}
\expandafter\def\csname PY@tok@mb\endcsname{\def\PY@tc##1{\textcolor[rgb]{0.40,0.40,0.40}{##1}}}
\expandafter\def\csname PY@tok@mf\endcsname{\def\PY@tc##1{\textcolor[rgb]{0.40,0.40,0.40}{##1}}}
\expandafter\def\csname PY@tok@mh\endcsname{\def\PY@tc##1{\textcolor[rgb]{0.40,0.40,0.40}{##1}}}
\expandafter\def\csname PY@tok@mi\endcsname{\def\PY@tc##1{\textcolor[rgb]{0.40,0.40,0.40}{##1}}}
\expandafter\def\csname PY@tok@il\endcsname{\def\PY@tc##1{\textcolor[rgb]{0.40,0.40,0.40}{##1}}}
\expandafter\def\csname PY@tok@mo\endcsname{\def\PY@tc##1{\textcolor[rgb]{0.40,0.40,0.40}{##1}}}
\expandafter\def\csname PY@tok@ch\endcsname{\let\PY@it=\textit\def\PY@tc##1{\textcolor[rgb]{0.25,0.50,0.50}{##1}}}
\expandafter\def\csname PY@tok@cm\endcsname{\let\PY@it=\textit\def\PY@tc##1{\textcolor[rgb]{0.25,0.50,0.50}{##1}}}
\expandafter\def\csname PY@tok@cpf\endcsname{\let\PY@it=\textit\def\PY@tc##1{\textcolor[rgb]{0.25,0.50,0.50}{##1}}}
\expandafter\def\csname PY@tok@c1\endcsname{\let\PY@it=\textit\def\PY@tc##1{\textcolor[rgb]{0.25,0.50,0.50}{##1}}}
\expandafter\def\csname PY@tok@cs\endcsname{\let\PY@it=\textit\def\PY@tc##1{\textcolor[rgb]{0.25,0.50,0.50}{##1}}}

\def\PYZbs{\char`\\}
\def\PYZus{\char`\_}
\def\PYZob{\char`\{}
\def\PYZcb{\char`\}}
\def\PYZca{\char`\^}
\def\PYZam{\char`\&}
\def\PYZlt{\char`\<}
\def\PYZgt{\char`\>}
\def\PYZsh{\char`\#}
\def\PYZpc{\char`\%}
\def\PYZdl{\char`\$}
\def\PYZhy{\char`\-}
\def\PYZsq{\char`\'}
\def\PYZdq{\char`\"}
\def\PYZti{\char`\~}
% for compatibility with earlier versions
\def\PYZat{@}
\def\PYZlb{[}
\def\PYZrb{]}
\makeatother


    % Exact colors from NB
    \definecolor{incolor}{rgb}{0.0, 0.0, 0.5}
    \definecolor{outcolor}{rgb}{0.545, 0.0, 0.0}



    
    % Prevent overflowing lines due to hard-to-break entities
    \sloppy 
    % Setup hyperref package
    \hypersetup{
      breaklinks=true,  % so long urls are correctly broken across lines
      colorlinks=true,
      urlcolor=urlcolor,
      linkcolor=linkcolor,
      citecolor=citecolor,
      }
    % Slightly bigger margins than the latex defaults
    
    \geometry{verbose,tmargin=1in,bmargin=1in,lmargin=1in,rmargin=1in}
    
    

    \begin{document}
    
    
    \maketitle
    
    

    
    \subsection{Previsão de Sobrevivência do
Titanic}\label{previsuxe3o-de-sobrevivuxeancia-do-titanic}

Vamos analisar o conjunto de dado que contém informações e dados
demográficos de 891 dentre os 2.224 passageiros e tripulação a bordo do
Titanic e tentar descobrir o motivo de alguns grupos de pessoas serem
mais propensas a sobreviver do que outras. Para fazer isso, iremos
seguir os seguintes passos:

\begin{itemize}
\tightlist
\item
  Análise e exploração dos dados
\item
  Brainstorm
\item
  Wrangle(Limpar, preparar e arrumar os dados)
\item
  Visualizar, reportar e demonstrar nossas descobertas
\end{itemize}

    \subsection{1. Importando bibliotecas
essencias}\label{importando-bibliotecas-essencias}

\begin{itemize}
\tightlist
\item
  Pandas: Utilizada para transformar os dados contidos no csv em um
  dataframe de fácil manipulação
\item
  Numpy : Utilizada para facilitar nosso esforço na análise dos dados
\item
  Seaborn e Matplotlib : Utilizada para construir, customizar e mostrar
  os gráficos da análise feita
\end{itemize}

    \begin{Verbatim}[commandchars=\\\{\}]
{\color{incolor}In [{\color{incolor}616}]:} \PY{c+c1}{\PYZsh{}Importing the pandas library, responsable to manipulate and tranform the data contained on csv files on dataframes}
          \PY{k+kn}{import} \PY{n+nn}{pandas} \PY{k}{as} \PY{n+nn}{pd}
          \PY{c+c1}{\PYZsh{}Importing the numpy library, responsible to smooth our effort in analyze the data}
          \PY{k+kn}{import} \PY{n+nn}{numpy} \PY{k}{as} \PY{n+nn}{np}
          \PY{c+c1}{\PYZsh{}Importing the matplotlib and seaborn libraries, responsable to show the graphs about the findings}
          \PY{k+kn}{import} \PY{n+nn}{seaborn} \PY{k}{as} \PY{n+nn}{sns}
          \PY{k+kn}{import} \PY{n+nn}{matplotlib}\PY{n+nn}{.}\PY{n+nn}{pyplot} \PY{k}{as} \PY{n+nn}{plt}
          
          \PY{o}{\PYZpc{}}\PY{k}{matplotlib} inline
\end{Verbatim}


    \subsection{2. Carregando os dados e verificando o seu
conteúdo}\label{carregando-os-dados-e-verificando-o-seu-conteuxfado}

Pandas é muito útil para carregar e trabalhar com os dados contidos no
arquivo csv. Aqui estamos carregando os dados contidos em um csv para
uma estrutura do Panda chamada dataframe do Panda de fácil manipulação e
também carregando uma parte dos dados para verificar seu conteúdo.

    \begin{Verbatim}[commandchars=\\\{\}]
{\color{incolor}In [{\color{incolor}617}]:} \PY{n}{passengers}  \PY{o}{=} \PY{n}{pd}\PY{o}{.}\PY{n}{read\PYZus{}csv}\PY{p}{(}\PY{l+s+s1}{\PYZsq{}}\PY{l+s+s1}{titanic\PYZhy{}data\PYZhy{}6.csv}\PY{l+s+s1}{\PYZsq{}}\PY{p}{)}
          \PY{n}{passengers}\PY{o}{.}\PY{n}{head}\PY{p}{(}\PY{p}{)}
\end{Verbatim}


\begin{Verbatim}[commandchars=\\\{\}]
{\color{outcolor}Out[{\color{outcolor}617}]:}    PassengerId  Survived  Pclass  \textbackslash{}
          0            1         0       3   
          1            2         1       1   
          2            3         1       3   
          3            4         1       1   
          4            5         0       3   
          
                                                          Name     Sex   Age  SibSp  \textbackslash{}
          0                            Braund, Mr. Owen Harris    male  22.0      1   
          1  Cumings, Mrs. John Bradley (Florence Briggs Th{\ldots}  female  38.0      1   
          2                             Heikkinen, Miss. Laina  female  26.0      0   
          3       Futrelle, Mrs. Jacques Heath (Lily May Peel)  female  35.0      1   
          4                           Allen, Mr. William Henry    male  35.0      0   
          
             Parch            Ticket     Fare Cabin Embarked  
          0      0         A/5 21171   7.2500   NaN        S  
          1      0          PC 17599  71.2833   C85        C  
          2      0  STON/O2. 3101282   7.9250   NaN        S  
          3      0            113803  53.1000  C123        S  
          4      0            373450   8.0500   NaN        S  
\end{Verbatim}
            
    \subsection{3. Verificando algumas
estatísticas}\label{verificando-algumas-estatuxedsticas}

\begin{itemize}
\tightlist
\item
  Total de passageiros é de 891 no total
\item
  A taxa de sobrevivência foi em torno de 38\%
\item
  Por volta de 75\% dos passageiros viajaram sozinhos
\item
  A média de idade é perto de 30 anos
\item
  A média da tarifa paga foi em torno de 32
\end{itemize}

    \begin{Verbatim}[commandchars=\\\{\}]
{\color{incolor}In [{\color{incolor}618}]:} \PY{n}{passengers}\PY{o}{.}\PY{n}{describe}\PY{p}{(}\PY{p}{)}
\end{Verbatim}


\begin{Verbatim}[commandchars=\\\{\}]
{\color{outcolor}Out[{\color{outcolor}618}]:}        PassengerId    Survived      Pclass         Age       SibSp  \textbackslash{}
          count   891.000000  891.000000  891.000000  714.000000  891.000000   
          mean    446.000000    0.383838    2.308642   29.699118    0.523008   
          std     257.353842    0.486592    0.836071   14.526497    1.102743   
          min       1.000000    0.000000    1.000000    0.420000    0.000000   
          25\%     223.500000    0.000000    2.000000   20.125000    0.000000   
          50\%     446.000000    0.000000    3.000000   28.000000    0.000000   
          75\%     668.500000    1.000000    3.000000   38.000000    1.000000   
          max     891.000000    1.000000    3.000000   80.000000    8.000000   
          
                      Parch        Fare  
          count  891.000000  891.000000  
          mean     0.381594   32.204208  
          std      0.806057   49.693429  
          min      0.000000    0.000000  
          25\%      0.000000    7.910400  
          50\%      0.000000   14.454200  
          75\%      0.000000   31.000000  
          max      6.000000  512.329200  
\end{Verbatim}
            
    \subsection{4. Formulando a pergunta}\label{formulando-a-pergunta}

Após conhecer e entender a estrutura dos nossos dados, assim como os
seus valores, a dúvida que mais se acentua é :

\emph{Quais features(colunas) tem uma correlação com o fator
sobrevivência?}

Como em todo o desastre, preferências de sobrevivência são dadas
moralmente para mulheres, crianças, idosos e familias, porém também
sabemos que pessoas de alto poder aquisitivo podem ter sido
beneficiadas.

Então vamos analisar e descobrir se os fatores como
\emph{Taxa(FareGroup)},\emph{Classe do Ticket(Pclass)},
\emph{Genêro(Sex)}, \emph{Família/Sozinho(FamilySize)} afetaram o fator
sobrevivência também no desastre do Titanic.

As perguntas seram as seguintes:

\begin{itemize}
\tightlist
\item
  O fator classe ou fator taxa da passagem tiveram relação com o fator
  sobrevivência? Pessoas com poder aquisitivo maior tinham propensão a
  sobreviver mais que pessoas com poder aquisitivo menor?
\item
  O fator ser passageiro único ou viajar em família teve relação com o
  fator sobrevivência? Famílias tiveram mais chances de sobreviver do
  que pessoas viajando sozinha? E o tamanho da família interferia nessa
  correlação?
\item
  O fator genêro teve relação com o fator sobrevivência? Mulheres
  tiveram mais propensão a sobreviver do que os homens?
\end{itemize}

    \subsection{5. Classificando as colunas do
dataframe}\label{classificando-as-colunas-do-dataframe}

\textbf{Dados categóricos}

São dados que servem de label de um grupo de items ou indivíduos. Dados
categóricos possuem a seguinte subdivisão: * \textbf{Ordinal: } São
dados que não ordem ou algum tipo de ranking de classificação associado.
* \textbf{Nominal: } São dados que possuem um ranking de classificação
associado.

\textbf{Dados Quantitativos}

São dados na qual podemos operar matematicamente em cima deles para
obter insights úteis. Dados quantitativos possuem a seguinte subdivisão:
* \textbf{Discreto: } São dados que podem ser subdivididos cada vez mais
em menores unidades. * \textbf{Contínuo: } São dados que não podem ser
subdividios em menores unidades.

Nossa classificação das colunas é a seguinte:

\textbf{Dados categóricos}

\begin{itemize}
\tightlist
\item
  \textbf{Ordinal: } Pclass
\item
  \textbf{Nominal: } Survived, Sex, and Embarked
\end{itemize}

\textbf{Dados Quantitativos}

\begin{itemize}
\tightlist
\item
  \textbf{Discreto: } SibSp, Parch
\item
  \textbf{Contínuo: } Age, Fare
\end{itemize}

    \subsection{6. Verificando a qualidades dos
dados}\label{verificando-a-qualidades-dos-dados}

Realizaremos uma pré-análise para descobrir se existem colunas com dados
faltantes, nulos ou vazio.

Podemos verificar que as colunas Age, Cabin and Embarked possuem 177,
687 e 2 dados nulos respectivamentos.

    \begin{Verbatim}[commandchars=\\\{\}]
{\color{incolor}In [{\color{incolor}619}]:} \PY{n+nb}{print}\PY{p}{(}\PY{n}{passengers}\PY{o}{.}\PY{n}{isnull}\PY{p}{(}\PY{p}{)}\PY{o}{.}\PY{n}{sum}\PY{p}{(}\PY{p}{)}\PY{p}{)}
          \PY{n+nb}{print}\PY{p}{(}\PY{n}{passengers}\PY{o}{.}\PY{n}{info}\PY{p}{(}\PY{p}{)}\PY{p}{)}
          \PY{n+nb}{print}\PY{p}{(}\PY{l+s+s2}{\PYZdq{}}\PY{l+s+s2}{Duplicated \PYZhy{} }\PY{l+s+si}{\PYZob{}\PYZcb{}}\PY{l+s+s2}{\PYZdq{}}\PY{o}{.}\PY{n}{format}\PY{p}{(}\PY{n}{passengers}\PY{o}{.}\PY{n}{duplicated}\PY{p}{(}\PY{p}{)}\PY{o}{.}\PY{n}{sum}\PY{p}{(}\PY{p}{)}\PY{p}{)}\PY{p}{)}
\end{Verbatim}


    \begin{Verbatim}[commandchars=\\\{\}]
PassengerId      0
Survived         0
Pclass           0
Name             0
Sex              0
Age            177
SibSp            0
Parch            0
Ticket           0
Fare             0
Cabin          687
Embarked         2
dtype: int64
<class 'pandas.core.frame.DataFrame'>
RangeIndex: 891 entries, 0 to 890
Data columns (total 12 columns):
PassengerId    891 non-null int64
Survived       891 non-null int64
Pclass         891 non-null int64
Name           891 non-null object
Sex            891 non-null object
Age            714 non-null float64
SibSp          891 non-null int64
Parch          891 non-null int64
Ticket         891 non-null object
Fare           891 non-null float64
Cabin          204 non-null object
Embarked       889 non-null object
dtypes: float64(2), int64(5), object(5)
memory usage: 83.6+ KB
None
Duplicated - 0

    \end{Verbatim}

    \subsection{7. Estratégia para as colunas com dados
faltantes}\label{estratuxe9gia-para-as-colunas-com-dados-faltantes}

\subsubsection{7.1 Dados Quantitativos}\label{dados-quantitativos}

Iremos complementar as colunas com dados faltantes utilizando a mediana
pra cada Title e transformar o tipo da coluna pra inteiro, que é a
representação adequada para a coluna. Os seguintes passos serão
executados: 1. Criaremos uma coluna chamada Title para nos ajudar a
calcular a idade referente ao título que a pessoa possui no nome 2.
Calcularemos a mediana da idade pra cada título 3. Preencheremos os
valores nulos com a mediana da idade referente ao título que a pessoa
possui

    \begin{Verbatim}[commandchars=\\\{\}]
{\color{incolor}In [{\color{incolor}620}]:} \PY{n}{passengers}\PY{p}{[}\PY{l+s+s2}{\PYZdq{}}\PY{l+s+s2}{Title}\PY{l+s+s2}{\PYZdq{}}\PY{p}{]} \PY{o}{=} \PY{n}{passengers}\PY{o}{.}\PY{n}{Name}\PY{o}{.}\PY{n}{str}\PY{o}{.}\PY{n}{extract}\PY{p}{(}\PY{l+s+s1}{\PYZsq{}}\PY{l+s+s1}{([A\PYZhy{}Za\PYZhy{}z]+)}\PY{l+s+s1}{\PYZbs{}}\PY{l+s+s1}{.}\PY{l+s+s1}{\PYZsq{}}\PY{p}{,} \PY{n}{expand}\PY{o}{=}\PY{k+kc}{False}\PY{p}{)}
          \PY{n}{ages}                \PY{o}{=} \PY{n}{passengers}\PY{o}{.}\PY{n}{groupby}\PY{p}{(}\PY{l+s+s1}{\PYZsq{}}\PY{l+s+s1}{Title}\PY{l+s+s1}{\PYZsq{}}\PY{p}{)}\PY{p}{[}\PY{l+s+s1}{\PYZsq{}}\PY{l+s+s1}{Age}\PY{l+s+s1}{\PYZsq{}}\PY{p}{]}\PY{o}{.}\PY{n}{median}\PY{p}{(}\PY{p}{)}
          \PY{k}{for} \PY{n}{title} \PY{o+ow}{in} \PY{n}{passengers}\PY{o}{.}\PY{n}{Title}\PY{o}{.}\PY{n}{unique}\PY{p}{(}\PY{p}{)}\PY{p}{:}
              \PY{n}{median} \PY{o}{=} \PY{n}{ages}\PY{p}{[}\PY{n}{title}\PY{p}{]}
              \PY{n}{passengers}\PY{o}{.}\PY{n}{loc}\PY{p}{[}\PY{p}{(}\PY{n}{passengers}\PY{o}{.}\PY{n}{Age}\PY{o}{.}\PY{n}{isnull}\PY{p}{(}\PY{p}{)}\PY{p}{)} \PY{o}{\PYZam{}} \PY{p}{(}\PY{n}{passengers}\PY{o}{.}\PY{n}{Title} \PY{o}{==} \PY{n}{title}\PY{p}{)}\PY{p}{,}\PY{l+s+s2}{\PYZdq{}}\PY{l+s+s2}{Age}\PY{l+s+s2}{\PYZdq{}}\PY{p}{]} \PY{o}{=} \PY{n}{median}
          
          \PY{n}{passengers}\PY{o}{.}\PY{n}{info}\PY{p}{(}\PY{p}{)}
\end{Verbatim}


    \begin{Verbatim}[commandchars=\\\{\}]
<class 'pandas.core.frame.DataFrame'>
RangeIndex: 891 entries, 0 to 890
Data columns (total 13 columns):
PassengerId    891 non-null int64
Survived       891 non-null int64
Pclass         891 non-null int64
Name           891 non-null object
Sex            891 non-null object
Age            891 non-null float64
SibSp          891 non-null int64
Parch          891 non-null int64
Ticket         891 non-null object
Fare           891 non-null float64
Cabin          204 non-null object
Embarked       889 non-null object
Title          891 non-null object
dtypes: float64(2), int64(5), object(6)
memory usage: 90.6+ KB

    \end{Verbatim}

    \subsection{7.2 Dados Qualitativos}\label{dados-qualitativos}

Iremos completar as colunas com os dados faltantes utilizando a moda da
distribuição. A coluna cabine possui vários valores nulos e como já
temos a classe do ticket não irá nos atrapalhar na análise, então iremos
dropar essa coluna.

    \begin{Verbatim}[commandchars=\\\{\}]
{\color{incolor}In [{\color{incolor}621}]:} \PY{n}{passengers}\PY{o}{.}\PY{n}{Embarked}\PY{o}{.}\PY{n}{fillna}\PY{p}{(}\PY{n}{passengers}\PY{p}{[}\PY{l+s+s1}{\PYZsq{}}\PY{l+s+s1}{Embarked}\PY{l+s+s1}{\PYZsq{}}\PY{p}{]}\PY{o}{.}\PY{n}{mode}\PY{p}{(}\PY{p}{)}\PY{p}{[}\PY{l+m+mi}{0}\PY{p}{]}\PY{p}{,}\PY{n}{inplace}\PY{o}{=}\PY{k+kc}{True}\PY{p}{)}
          \PY{n}{passengers}\PY{o}{.}\PY{n}{drop}\PY{p}{(}\PY{l+s+s2}{\PYZdq{}}\PY{l+s+s2}{Cabin}\PY{l+s+s2}{\PYZdq{}}\PY{p}{,}\PY{n}{axis}\PY{o}{=}\PY{l+s+s2}{\PYZdq{}}\PY{l+s+s2}{columns}\PY{l+s+s2}{\PYZdq{}}\PY{p}{,}\PY{n}{inplace}\PY{o}{=}\PY{k+kc}{True}\PY{p}{)}
          \PY{n}{passengers}\PY{o}{.}\PY{n}{isnull}\PY{p}{(}\PY{p}{)}\PY{o}{.}\PY{n}{sum}\PY{p}{(}\PY{p}{)}
\end{Verbatim}


\begin{Verbatim}[commandchars=\\\{\}]
{\color{outcolor}Out[{\color{outcolor}621}]:} PassengerId    0
          Survived       0
          Pclass         0
          Name           0
          Sex            0
          Age            0
          SibSp          0
          Parch          0
          Ticket         0
          Fare           0
          Embarked       0
          Title          0
          dtype: int64
\end{Verbatim}
            
    \subsection{8. Criando novas colunas}\label{criando-novas-colunas}

Criaremos colunas novas afim de facilitar a análise dos dados

\begin{enumerate}
\def\labelenumi{\arabic{enumi}.}
\tightlist
\item
  FamilySize - Irá representar a quantidade de pessoas de uma família na
  qual essa pessoa pertence. Se o valor for 1, indica que essa pessoa
  viajou sozinha.
\item
  AgeGroup - Categorização das Idades.
\item
  FareGroup - Categorização das Tarifas em 4 faixas
\end{enumerate}

    \begin{Verbatim}[commandchars=\\\{\}]
{\color{incolor}In [{\color{incolor}622}]:} \PY{n}{passengers}\PY{p}{[}\PY{l+s+s2}{\PYZdq{}}\PY{l+s+s2}{FamilySize}\PY{l+s+s2}{\PYZdq{}}\PY{p}{]}  \PY{o}{=} \PY{n}{passengers}\PY{o}{.}\PY{n}{SibSp} \PY{o}{+} \PY{n}{passengers}\PY{o}{.}\PY{n}{Parch}\PY{o}{+} \PY{l+m+mi}{1}
          \PY{n}{passengers}\PY{o}{.}\PY{n}{Age}            \PY{o}{=} \PY{n}{passengers}\PY{o}{.}\PY{n}{Age}\PY{o}{.}\PY{n}{astype}\PY{p}{(}\PY{n+nb}{int}\PY{p}{)}
          \PY{n}{passengers}\PY{p}{[}\PY{l+s+s2}{\PYZdq{}}\PY{l+s+s2}{AgeGroup}\PY{l+s+s2}{\PYZdq{}}\PY{p}{]}    \PY{o}{=} \PY{n}{pd}\PY{o}{.}\PY{n}{cut}\PY{p}{(}\PY{n}{passengers}\PY{o}{.}\PY{n}{Age}\PY{p}{,}\PY{n+nb}{range}\PY{p}{(}\PY{l+m+mi}{0}\PY{p}{,}\PY{l+m+mi}{90}\PY{p}{,}\PY{l+m+mi}{15}\PY{p}{)}\PY{p}{)}
          \PY{n}{passengers}\PY{p}{[}\PY{l+s+s2}{\PYZdq{}}\PY{l+s+s2}{FareGroup}\PY{l+s+s2}{\PYZdq{}}\PY{p}{]}   \PY{o}{=} \PY{n}{pd}\PY{o}{.}\PY{n}{cut}\PY{p}{(}\PY{n}{passengers}\PY{o}{.}\PY{n}{Fare}\PY{p}{,}\PY{l+m+mi}{4}\PY{p}{,}\PY{n}{labels}\PY{o}{=}\PY{p}{[}\PY{l+s+s2}{\PYZdq{}}\PY{l+s+s2}{Cheap}\PY{l+s+s2}{\PYZdq{}}\PY{p}{,}\PY{l+s+s2}{\PYZdq{}}\PY{l+s+s2}{Normal}\PY{l+s+s2}{\PYZdq{}}\PY{p}{,}\PY{l+s+s2}{\PYZdq{}}\PY{l+s+s2}{Expensive}\PY{l+s+s2}{\PYZdq{}}\PY{p}{,}\PY{l+s+s2}{\PYZdq{}}\PY{l+s+s2}{Very Expensive}\PY{l+s+s2}{\PYZdq{}}\PY{p}{]}\PY{p}{)}
          \PY{n}{passengers}\PY{o}{.}\PY{n}{FamilySize}     \PY{o}{=} \PY{n}{pd}\PY{o}{.}\PY{n}{cut}\PY{p}{(}\PY{n}{passengers}\PY{o}{.}\PY{n}{FamilySize}\PY{p}{,}\PY{p}{[}\PY{l+m+mi}{0}\PY{p}{,}\PY{l+m+mi}{1}\PY{p}{,}\PY{l+m+mi}{4}\PY{p}{,}\PY{n}{passengers}\PY{o}{.}\PY{n}{FamilySize}\PY{o}{.}\PY{n}{max}\PY{p}{(}\PY{p}{)}\PY{p}{]}\PY{p}{,}\PY{n}{labels}\PY{o}{=}\PY{p}{[}\PY{l+s+s2}{\PYZdq{}}\PY{l+s+s2}{Single}\PY{l+s+s2}{\PYZdq{}}\PY{p}{,}\PY{l+s+s2}{\PYZdq{}}\PY{l+s+s2}{Normal}\PY{l+s+s2}{\PYZdq{}}\PY{p}{,}\PY{l+s+s2}{\PYZdq{}}\PY{l+s+s2}{Large}\PY{l+s+s2}{\PYZdq{}}\PY{p}{]}\PY{p}{)}
\end{Verbatim}


    \subsection{9. Visualização Features x
Sobrevivência}\label{visualizauxe7uxe3o-features-x-sobrevivuxeancia}

Primeiro construímos uma função que passado o nome da feature, ela gera
um gráfico relacionado a feature Survived

    \begin{Verbatim}[commandchars=\\\{\}]
{\color{incolor}In [{\color{incolor}623}]:} \PY{k}{def} \PY{n+nf}{generate\PYZus{}graph}\PY{p}{(}\PY{n}{feature}\PY{p}{,}\PY{n}{graph}\PY{o}{=}\PY{l+s+s2}{\PYZdq{}}\PY{l+s+s2}{bar}\PY{l+s+s2}{\PYZdq{}}\PY{p}{)}\PY{p}{:}
              \PY{l+s+sd}{\PYZdq{}\PYZdq{}\PYZdq{}}
          \PY{l+s+sd}{    Função que recebe o nome da feature e gera o gráfico relacionado a feature Survived}
          \PY{l+s+sd}{    Argumentos:}
          \PY{l+s+sd}{        feature:  String contendo o nome da feature}
          \PY{l+s+sd}{    Retorna:}
          \PY{l+s+sd}{        None}
          \PY{l+s+sd}{    \PYZdq{}\PYZdq{}\PYZdq{}}
              \PY{n}{ax}       \PY{o}{=} \PY{k+kc}{None}
              \PY{n}{group}    \PY{o}{=} \PY{n}{passengers}\PY{o}{.}\PY{n}{groupby}\PY{p}{(}\PY{p}{[}\PY{l+s+s2}{\PYZdq{}}\PY{l+s+s2}{Survived}\PY{l+s+s2}{\PYZdq{}}\PY{p}{,}\PY{n}{feature}\PY{p}{]}\PY{p}{,}\PY{n}{sort}\PY{o}{=}\PY{k+kc}{True}\PY{p}{)}\PY{o}{.}\PY{n}{size}\PY{p}{(}\PY{p}{)}
              \PY{n}{df\PYZus{}aux}   \PY{o}{=} \PY{n}{pd}\PY{o}{.}\PY{n}{concat}\PY{p}{(}\PY{p}{[}\PY{n}{group}\PY{p}{[}\PY{l+m+mi}{0}\PY{p}{]}\PY{p}{,}\PY{n}{group}\PY{p}{[}\PY{l+m+mi}{1}\PY{p}{]}\PY{p}{]}\PY{p}{,}\PY{n}{axis}\PY{o}{=}\PY{l+m+mi}{1}\PY{p}{)}\PY{o}{.}\PY{n}{fillna}\PY{p}{(}\PY{l+m+mi}{0}\PY{p}{)}
              \PY{n}{ax}       \PY{o}{=} \PY{n}{df\PYZus{}aux}\PY{o}{.}\PY{n}{plot}\PY{p}{(}\PY{n}{kind}\PY{o}{=}\PY{n}{graph}\PY{p}{,}\PY{n}{title}\PY{o}{=}\PY{n}{feature}\PY{p}{,}\PY{n}{grid}\PY{o}{=}\PY{k+kc}{True}\PY{p}{,}\PY{n}{legend}\PY{o}{=}\PY{l+s+s2}{\PYZdq{}}\PY{l+s+s2}{reverse}\PY{l+s+s2}{\PYZdq{}}\PY{p}{,}\PY{n}{rot}\PY{o}{=}\PY{l+m+mi}{60}\PY{p}{,}\PY{n}{stacked}\PY{o}{=}\PY{k+kc}{False}\PY{p}{)}
              \PY{n}{ax}\PY{o}{.}\PY{n}{set\PYZus{}ylabel}\PY{p}{(}\PY{l+s+s2}{\PYZdq{}}\PY{l+s+s2}{Number of people}\PY{l+s+s2}{\PYZdq{}}\PY{p}{)}
              \PY{n}{ax}\PY{o}{.}\PY{n}{legend}\PY{p}{(}\PY{p}{[}\PY{l+s+s2}{\PYZdq{}}\PY{l+s+s2}{Not Survived}\PY{l+s+s2}{\PYZdq{}}\PY{p}{,} \PY{l+s+s2}{\PYZdq{}}\PY{l+s+s2}{Survived}\PY{l+s+s2}{\PYZdq{}}\PY{p}{]}\PY{p}{)}\PY{p}{;}
              
\end{Verbatim}


    \subsection{9.1 Gráfico Pclass x
Survived}\label{gruxe1fico-pclass-x-survived}

No gráfico vemos que a 1º classe foi a única com taxa de sobrevivência
maior do que a de não sobrevivência comparada as outras classes(2º,3º) e
que na 3º classe a discrepância entre mortos e sobreviventes foi bem
alta comparada as outras classes(1º,2º), com muito mais mortos do que
sobreviventes.

A tendência de sobrevivência é maior em pessoas da 1º mas no caso das
outras classes(2º,3º) principalmente na 3º, a tendência é de não
sobrevivência

    \begin{Verbatim}[commandchars=\\\{\}]
{\color{incolor}In [{\color{incolor}624}]:} \PY{n}{generate\PYZus{}graph}\PY{p}{(}\PY{l+s+s2}{\PYZdq{}}\PY{l+s+s2}{Pclass}\PY{l+s+s2}{\PYZdq{}}\PY{p}{)}
\end{Verbatim}


    \begin{center}
    \adjustimage{max size={0.9\linewidth}{0.9\paperheight}}{output_20_0.png}
    \end{center}
    { \hspace*{\fill} \\}
    
    \subsection{9.2 Gráfico Sex x Survived}\label{gruxe1fico-sex-x-survived}

No gráfico vemos que a taxa de sobrevivência de mulheres foi maior do
que a de não sobrevivência comparada a taxa dos homens.

A tendência de sobrevivência é maior nas mulheres do que nos homens.

    \begin{Verbatim}[commandchars=\\\{\}]
{\color{incolor}In [{\color{incolor}625}]:} \PY{n}{generate\PYZus{}graph}\PY{p}{(}\PY{l+s+s2}{\PYZdq{}}\PY{l+s+s2}{Sex}\PY{l+s+s2}{\PYZdq{}}\PY{p}{)}
\end{Verbatim}


    \begin{center}
    \adjustimage{max size={0.9\linewidth}{0.9\paperheight}}{output_22_0.png}
    \end{center}
    { \hspace*{\fill} \\}
    
    \subsection{9.3 Gráfico FamilySize x
Survived}\label{gruxe1fico-familysize-x-survived}

Temos uma categorização da quantidade de familiares: * Single - (1) *
Normal - (2 - 4) * Large - (5 - 11)

No gráfico vemos famílias de tamanho normal, de 2 até 4 integrantes foi
a única com taxa de sobrevivência maior do que a de não sobrevivência
comparada com as outras categorias.

A tendência de sobrevivência é maior em famílias de tamanho normal, do
que em famílias grandes ou até mesmo se a pessoa estivesse viajando
sozinha.

    \begin{Verbatim}[commandchars=\\\{\}]
{\color{incolor}In [{\color{incolor}626}]:} \PY{n}{generate\PYZus{}graph}\PY{p}{(}\PY{l+s+s2}{\PYZdq{}}\PY{l+s+s2}{FamilySize}\PY{l+s+s2}{\PYZdq{}}\PY{p}{)}
\end{Verbatim}


    \begin{center}
    \adjustimage{max size={0.9\linewidth}{0.9\paperheight}}{output_24_0.png}
    \end{center}
    { \hspace*{\fill} \\}
    
    \subsection{9.4 Gráfico FareGroup x Survived e Fare x
Survived}\label{gruxe1fico-faregroup-x-survived-e-fare-x-survived}

Temos uma categorização dos valores da Tarifa pagas: * Cheap - (0 -
128.082) * Normal - (128.083 - 256.165) * Expensive - (256.166 -
384.247) * Very Expensive - (384.248 - 512.329)

No gráfico vemos que a tarifa mais barata foi a única que teve a taxa de
não sobrevivência maior do que de sobrevivência e com uma discrepância
muito alta comparada com as outras tarifas.

A tendência de sobrevivência é maior em pessoas que obtiveram taxas
entre Normal e Carissimas(Normal - Very Expensive) comparadas as taxas
mais baratas(Cheap).

    \begin{Verbatim}[commandchars=\\\{\}]
{\color{incolor}In [{\color{incolor}627}]:} \PY{n}{generate\PYZus{}graph}\PY{p}{(}\PY{l+s+s2}{\PYZdq{}}\PY{l+s+s2}{FareGroup}\PY{l+s+s2}{\PYZdq{}}\PY{p}{)}
          \PY{n}{generate\PYZus{}graph}\PY{p}{(}\PY{l+s+s2}{\PYZdq{}}\PY{l+s+s2}{Fare}\PY{l+s+s2}{\PYZdq{}}\PY{p}{,}\PY{l+s+s2}{\PYZdq{}}\PY{l+s+s2}{area}\PY{l+s+s2}{\PYZdq{}}\PY{p}{)}
\end{Verbatim}


    \begin{center}
    \adjustimage{max size={0.9\linewidth}{0.9\paperheight}}{output_26_0.png}
    \end{center}
    { \hspace*{\fill} \\}
    
    \begin{center}
    \adjustimage{max size={0.9\linewidth}{0.9\paperheight}}{output_26_1.png}
    \end{center}
    { \hspace*{\fill} \\}
    
    \subsection{9.4 Gráfico AgeGroup x Survived e Age x
Survived}\label{gruxe1fico-agegroup-x-survived-e-age-x-survived}

No gráfico vemos que a faixa etária de (0 - 15) foi a única que teve
mais sobreviventes do que mortos e que na faixa de (15 - 30) a
discrepância entre mortos e sobreviventes foi bem alta comparada as
outras faixas, com muito mais mortos do que sobreviventes.

A tendência de sobrevivência é maior em pessoas de 0 até 15 anos mas no
caso das outras faixas etárias principalmente de 15 até 30, a tendência
é de não sobrevivência

    \begin{Verbatim}[commandchars=\\\{\}]
{\color{incolor}In [{\color{incolor}628}]:} \PY{n}{generate\PYZus{}graph}\PY{p}{(}\PY{l+s+s2}{\PYZdq{}}\PY{l+s+s2}{AgeGroup}\PY{l+s+s2}{\PYZdq{}}\PY{p}{)}
          \PY{n}{generate\PYZus{}graph}\PY{p}{(}\PY{l+s+s2}{\PYZdq{}}\PY{l+s+s2}{Age}\PY{l+s+s2}{\PYZdq{}}\PY{p}{,}\PY{l+s+s2}{\PYZdq{}}\PY{l+s+s2}{area}\PY{l+s+s2}{\PYZdq{}}\PY{p}{)}
\end{Verbatim}


    \begin{center}
    \adjustimage{max size={0.9\linewidth}{0.9\paperheight}}{output_28_0.png}
    \end{center}
    { \hspace*{\fill} \\}
    
    \begin{center}
    \adjustimage{max size={0.9\linewidth}{0.9\paperheight}}{output_28_1.png}
    \end{center}
    { \hspace*{\fill} \\}
    
    \subsection{10 - Lista de sites, livros, fóruns, postagens de blogs,
repositórios do
GitHub.}\label{lista-de-sites-livros-fuxf3runs-postagens-de-blogs-reposituxf3rios-do-github.}

O arquivo links.txt contido no diretório, contém a listagem dos sites
que utilizei como base e ajuda para realizar esse projeto


    % Add a bibliography block to the postdoc
    
    
    
    \end{document}
